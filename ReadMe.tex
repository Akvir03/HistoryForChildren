% Options for packages loaded elsewhere
\PassOptionsToPackage{unicode}{hyperref}
\PassOptionsToPackage{hyphens}{url}
%
\documentclass[
]{article}
\usepackage{amsmath,amssymb}
\usepackage{iftex}
\ifPDFTeX
  \usepackage[T1]{fontenc}
  \usepackage[utf8]{inputenc}
  \usepackage{textcomp} % provide euro and other symbols
\else % if luatex or xetex
  \usepackage{unicode-math} % this also loads fontspec
  \defaultfontfeatures{Scale=MatchLowercase}
  \defaultfontfeatures[\rmfamily]{Ligatures=TeX,Scale=1}
\fi
\usepackage{lmodern}
\ifPDFTeX\else
  % xetex/luatex font selection
\fi
% Use upquote if available, for straight quotes in verbatim environments
\IfFileExists{upquote.sty}{\usepackage{upquote}}{}
\IfFileExists{microtype.sty}{% use microtype if available
  \usepackage[]{microtype}
  \UseMicrotypeSet[protrusion]{basicmath} % disable protrusion for tt fonts
}{}
\makeatletter
\@ifundefined{KOMAClassName}{% if non-KOMA class
  \IfFileExists{parskip.sty}{%
    \usepackage{parskip}
  }{% else
    \setlength{\parindent}{0pt}
    \setlength{\parskip}{6pt plus 2pt minus 1pt}}
}{% if KOMA class
  \KOMAoptions{parskip=half}}
\makeatother
\usepackage{xcolor}
\usepackage{graphicx}
\makeatletter
\def\maxwidth{\ifdim\Gin@nat@width>\linewidth\linewidth\else\Gin@nat@width\fi}
\def\maxheight{\ifdim\Gin@nat@height>\textheight\textheight\else\Gin@nat@height\fi}
\makeatother
% Scale images if necessary, so that they will not overflow the page
% margins by default, and it is still possible to overwrite the defaults
% using explicit options in \includegraphics[width, height, ...]{}
\setkeys{Gin}{width=\maxwidth,height=\maxheight,keepaspectratio}
% Set default figure placement to htbp
\makeatletter
\def\fps@figure{htbp}
\makeatother
\setlength{\emergencystretch}{3em} % prevent overfull lines
\providecommand{\tightlist}{%
  \setlength{\itemsep}{0pt}\setlength{\parskip}{0pt}}
\setcounter{secnumdepth}{-\maxdimen} % remove section numbering
\ifLuaTeX
  \usepackage{selnolig}  % disable illegal ligatures
\fi
\usepackage{bookmark}
\IfFileExists{xurl.sty}{\usepackage{xurl}}{} % add URL line breaks if available
\urlstyle{same}
\hypersetup{
  pdftitle={History for Children},
  pdfauthor={Nathan Devienne \textbar{} Paul Hurdebourcq \textbar{} Théo Massias \textbar{} Amzil Marwane},
  hidelinks,
  pdfcreator={LaTeX via pandoc}}

\title{History for Children}
\author{Nathan Devienne \textbar{} Paul Hurdebourcq \textbar{} Théo
Massias \textbar{} Amzil Marwane}
\date{December 2023}

\begin{document}
\maketitle

\section{Introduction}\label{introduction}

Le but de cet outil est de permettre à des jeunes enfants de découvrir
les mondes pré-historique, antiques et médiévaux de chez eux.

\section{Maquettes}\label{maquettes}

\subsection{Monde Préhistorique}\label{monde-pruxe9historique}

\begin{figure}
\centering
\includegraphics{https://raw.githubusercontent.com/Akvir03/HistoryForChildren/main/Documentation/Maquette/Pré-histoire\%201ere\%20personne.png}
\caption{Maquette préhistorique}
\end{figure}

\subsection{Monde Médiéval}\label{monde-muxe9diuxe9val}

\begin{figure}
\centering
\includegraphics{https://raw.githubusercontent.com/Akvir03/HistoryForChildren/main/Documentation/Maquette/Epoque\%20médiévale.png}
\caption{Maquette médiévale}
\end{figure}

\subsection{Monde Antique}\label{monde-antique}

\begin{figure}
\centering
\includegraphics{https://raw.githubusercontent.com/Akvir03/HistoryForChildren/main/Documentation/Maquette/Antiquité.png}
\caption{Maquette du monde antique}
\end{figure}

\section{Méthode d'Installation}\label{muxe9thode-dinstallation}

Pour installer ce projet, veuillez suivre les étapes suivantes :

\begin{enumerate}
\def\labelenumi{\arabic{enumi}.}
\item
  Téléchargez et installez UNITY à partir de
  \href{https://public-cdn.cloud.unity3d.com/hub/prod/UnityHubSetup.exe}{ce
  lien}.
\item
  Créez un compte Unity.
\item
  Téléchargez le projet depuis GitHub.
\item
  Ouvrez le projet dans Unity.
\item
  Lancez Steam et téléchargez Steam VR.
\item
  Connectez votre casque VR.
\item
  Appuyez sur le bouton PLAY dans Unity pour découvrir notre univers.
\end{enumerate}

\section{Personae}\label{personae}

Le public cible de ce projet est constitué de jeunes enfants âgés de 8 à
14 ans, curieux de découvrir les mondes préhistorique, médiéval et
antique.

\section{Crédits \& Modèles
utilisés}\label{cruxe9dits-moduxe8les-utilisuxe9s}

À venir

\end{document}
